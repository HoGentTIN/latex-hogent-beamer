%------------------------------------------------------------------------------
% Voorbeeld presentatie HoGent - huisstijl 2018
%------------------------------------------------------------------------------
\documentclass[aspectratio=169]{beamer}

% Je kan het lettertype iets vergroten door hierboven optie ``14pt'' toe te
% voegen.

%==============================================================================
% Aanloop
%==============================================================================

%---------- Vormgeving --------------------------------------------------------

\usetheme{hogent}

% Kies hieronder een achtergrondkleur
%\usecolortheme{hgwhite} % witte achtergrond, zwarte tekst
\usecolortheme{hgblack} % zwarte achtergrond, witte tekst

%---------- Packages ----------------------------------------------------------

\usepackage[utf8]{inputenc}    % Laat toe accenten in broncode te gebruiken
\usepackage[dutch]{babel}      % Nederlandse taal: splitsingen, enz.

\usepackage{booktabs}          % Mooie tabellen
\usepackage{multirow,multicol} % Tabelcellen samenvoegen
\usepackage{eurosym}           % Euro symbool

%---------- Commando-definities -----------------------------------------------

%---------- Info over de presentatie ------------------------------------------

\title[Korte titel]{Titel van de presentatie.}
\subtitle{subtitel}
\author[BVV]{Bert {Van Vreckem} (\href{mailto:bert.vanvreckem@hogent.be}{bert.vanvreckem@hogent.be})}
\date{\today}

%==============================================================================
% Inhoud presentatie
%==============================================================================

\begin{document}

\frame{\maketitle}

\begin{frame}
  \frametitle{Inhoud.}

  \tableofcontents
\end{frame}
 
%---------- Inhoud ------------------------------------------------------------

\section{Eerste deel.}

\begin{frame}
  \frametitle{Inleiding.}
\end{frame}

\subsection{Subsectie 1.1.}

\begin{frame}
  \frametitle{Titel.}

  \begin{itemize}
  \item Lijn één
  \item Lijn twee
  \item Lijn drie
  \end{itemize}
\end{frame}

\subsection{Subsectie 1.2.}

\begin{frame}
  \frametitle{Twee kolommen.}

  \begin{columns}[c]

  \column{.5\textwidth}
    \begin{itemize}
    \item Lijn 1
    \item Lijn 2
    \item Lijn 3
    \end{itemize}

  \column{.5\textwidth}
    \begin{itemize}
    \item Lijn 1
    \item Lijn 2
    \item Lijn 3
    \end{itemize}

  \end{columns}
\end{frame}

\section{Tweede deel.}

\begin{frame}
  \frametitle{Het tweede deel.}
  
  \begin{table}
    \begin{tabular}{lll}
      \toprule
      & \textbf{lorem ipsum} & \textbf{lorem ipsum} \\
      \midrule
      dolor sit amet & $3.255$ & $3.255$ \\
      \midrule
      lorem ipsum & $425,43 (13,07\%)$ & $425,43 (13,07\%)$ \\
      \midrule
      dolor sit amet & $0,00$ & $1.273,31$ ($45\%$ -- gemiddeld) \\
      \midrule
      \textbf{{\small Verhouding kost}} & \textbf{$1,53$} & \textbf{$2,72$}\\
      \bottomrule
    \end{tabular}
    
    \label{tab:voorbeeld}
    \caption{Deze tabel heeft een bijschrift}
  \end{table}
\end{frame}

\subsection{Sectie 2.1.}

\begin{frame}[fragile]
  \frametitle{Nadruk.}

  Voorbeeld van \alert{nadruk}
  


\end{frame}

\begin{frame}[fragile]
  \texttt{monogespatieerd lettertype met ligaturen -> <= >=}

\begin{verbatim}
a <- 1
b <- 2
a + b
\end{verbatim}
\end{frame}

%---------- Back matter -------------------------------------------------------

\end{document}
